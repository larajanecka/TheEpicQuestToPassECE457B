\subsection{Problem Statement}
Dance Dance Revolution (DDR) is a video game where the player attempts to press down arrows at specific times during a song, which simulates dance. A single DDR game consists of a music file and a steps file which are parsed and rendered by the game for playing. To generate a music file and steps file pair, an experienced person would need to manually map out the song to musical dance elements and indicate at which precise moments in time each arrow would need to be hit. The entire process would take 4 or more hours. An easier way to generate new playable songs is needed so that players can enjoy dancing to their favorite songs.\\

\subsection{Goal}
The goal of this project is to create a system that automatically generates an accompanying steps file with a music file as input. The system will be implemented using neural networks and fuzzy logic. Specifically, a  Training via supervised learning will occur using existing music files and steps files. The qualitative measurement of the goal is to assess the produced steps files by their playability, and whether or not they exhibits a touch of human creativity. The quantitative measurement of the goal is the percentage of the steps that are generated which occur at the same time as recognised musical features, for example beats.\\

\subsection{Neural Network Component}

The neural network model used to perform beat detection will be a standard feedforward network. Specifically, a custom implementation of a Multilayer Perceptron with variable number of layers and nodes per layer will be created. This model will also use the sigmoid activation function and the online-updating rule for backpropagation. A custom implementation was desired in order to give us experience at implementing a feedforward network.\\

Also, a third party library will be used to implement another feedforward network in order to give us two models to compare. This second model will be similar to the first, and be built using the third party Keras machine learning library.