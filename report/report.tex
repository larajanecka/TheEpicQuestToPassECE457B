\documentclass[12pt]{article}

\usepackage{fullpage,url,amssymb,epsfig,color,pdfpages,xspace,enumerate,multirow,enumitem,graphicx,amsmath}
\usepackage{listings}
\usepackage[pdftitle={ECE457b Project},%
pdfsubject={University of Waterloo, ECE 457b, Winter 2017},%
pdfauthor={Daniel Cardoza, Lara Janecka, Hong Wang}]{hyperref}

\begin{document}

\begin{center}
{\Large\bf Report for}\\
\vspace{1mm}
{\Large\bf ECE457B Course Project, Winter 2017}\\
\vspace{2mm}
{\Large\bf Ultimate DDR}\\
\vspace{4mm}
{Daniel Cardoza/dpmcardo/20471664}\\
{Lara Janecka/lajaneck/20460089}\\
{Hong Wang/hmwang/20469058}\\
\vspace{2mm}
\textbf{Due Date: April 3, 2017}
\end{center}

\definecolor{care}{rgb}{0,0,0}
\def\question#1{\item[\bf #1.]}
\def\part#1{\item[\bf #1)]}
\newcommand{\pc}[1]{\mbox{\textbf{#1}}} % pseudocode

\section{Abstract}
\subsection{Introduction}
The project proposed by the following abstract falls underneath Cluster 2 of the possible projects, \textbf{Pattern Recognition and
Perception}. The goal of Ultimate DR is to produce software that is able to generate Dance Dance Revolution (DDR) game
files using neural networks and fuzzy inferencing techniques.
DDR is a video game under the genre of rhythm games. In the game,
the player is presented with music and a sequence of arrows which they must press at an exact
time. Typically, the game is played on a foot-pad where the player presses 4 arrows with their
feet, which is akin to dancing to the music. Currently, DDR game files (SM files) are generated
manually by semi-professional or professional step makers. The step makers create the SM files
by listening to the song and creatively conjuring a set of steps to match the music; the better
the game steps matches the music, the more exciting the song is to play.


\subsection{Motivation}
The main problem with DDR as a game is the lack of SM files for a wide selection of songs. This
is due to the fact that to generate each SM file, a step maker will need to spend around 4 hours.
As the current demand for the game is not very high, the selection of songs for which step files
exists is very low (compared to number of existing songs). This is a major blocker for new
players to the game. A small study (outside the scope of this course) has been conducted to see
why people are not willing to try DDR or why people new to DDR have quit reveals that the one
driving factor to their patronage to the game is being able to play songs they love. Hence, if
software could be created to produce SM files from arbitrary songs, or arbitrary songs from a
genre set, it can easily revitalize the DDR community.

\subsection{Existing Data}
The data used for this project will be pairs of audio files along with the beat data for the files. Substantial data repositories in the field of audioengineering analysis exist that contain this data. A simple web scraping strategy or system can also be used to build this dataset. 

\subsection{Project Strategy}
Music files can always be decompressed to a wavelet or WAV file format. A WAV file holds a sequence of
integers which each denote a single sample amplitude of the music’s waveform. The project
implementation will first attempt to generate a set of music feature candidates by applying 
various Digital Signal Processing techniques and perhaps even some randomized (but guided)
functions to arbitrary samples of the waveform. Examples of music feature candidates include
possible beats, note changes, or any seemingly relevant patterns found when processing the
amplitude samples. The purpose of the music candidate feature generation is to create a broad set
of possible music features that the step file makers used to create the step file.\\

Fuzzy logic will
be implemented to determine whether or not a music feature candidate is indeed a relevant music
feature used to generate the steps file. Each identified beat in a song does not always map to a step in a step file and fuzzy logic can be used to determine when it is. A neural network will be used to facilitate the evaluation
of multiple features simultaneously, propagating the details of the feature if it is promising. The
neural network implementation is also expected to compare music feature candidates and
categorize them using fuzzy logic. The resulting neural network will then contain a rough
relationship of which music features found in a WAV file relates to steps in the steps file. It can
be then used to generate step files for a new WAV file.

\subsection{Drawbacks}

Potential drawbacks include having long algorithm running times. It is well known that some
DSP algorithms are slow (especially when operating over long audio samples like a song) and
adding feature candidate extracting will only increase run time. As well, this exact topic is not
well explored in the industry and overwhelming success has not yet been reported.
\pagebreak

\section{Introduction}
\subsection{Problem Statement}
Dance Dance Revolution (DDR) is a video game where the player attempts to press down arrows at specific times during a song, which simulates dance. A single DDR game consists of a music file and a steps file which are parsed and rendered by the game for playing. To generate a music file and steps file pair, an experienced person would need to manually map out the song to musical dance elements and indicate at which precise moments in time each arrow would need to be hit. The entire process would take 4 or more hours. An easier way to generate new playable songs is needed so that players can enjoy dancing to their favorite songs.\\

\subsection{Goal}
The goal of this project is to create a system that automatically generates an accompanying steps file with a music file as input. The system will be implemented using neural networks and fuzzy logic. Specifically, a  Training via supervised learning will occur using existing music files and steps files. The qualitative measurement of the goal is to assess the produced steps files by their playability, and whether or not they exhibits a touch of human creativity. The quantitative measurement of the goal is the percentage of the steps that are generated which occur at the same time as recognised musical features, for example beats.\\

\subsection{Neural Network Component}

The neural network model used to perform beat detection will be a standard feedforward network. Specifically, a custom implementation of a Multilayer Perceptron with variable number of layers and nodes per layer will be created. This model will also use the sigmoid activation function and the online-updating rule for backpropagation. A custom implementation was desired in order to give us experience at implementing a feedforward network.\\

Also, a third party library will be used to implement another feedforward network in order to give us two models to compare. This second model will be similar to the first, and be built using the third party Keras machine learning library.
\pagebreak


\section{Background}
\subsection{Music Files}
The files used for analysis were primarily wav files. To use another type of file required conversion into a wav file. Wav files consist of a header containing relevant information such as sample rate and the number of channels, and the sound data for each sample. This format does not compress the data and thus was chosen of the most accurate results. A parser was written to extract the sample rate and a waveform for the sound data. The sample rate is the number of samples taken per second. For CD quality (the most common type of file looked at) this value is 44,100. The waveform is the list of data for the sound data at each sample point for each channel.

\subsection{Stepmania Files}
\subsection{Overview}

Stepmania is a cross-platform dance and rhythm game. Players select a song and corresponding leve of difficulty when playing. Then, players view a screen where up, down, left and right arrows rise up on the screen. When the arrow reaches a certain point, the user clicks the corresponding arrow key on their keyboard.\\

Stepmania uses custom files containing necessary metadata and audio information. In the context of this project, each stepmania arrow corresponds to a different beat in the song being played. This makes the gaming experience more natural and enjoyable for the player. Our project seeks to perform beat detection, and then generate the corresponding Stepmania file with beats at the discrete times we detect. Note that a Stepmania file may only have steps for a subset of the beats in a song.\\

\subsection{File Format}

The Stepmania file format contains necessary header information indicating the corresponding song for the file as well as information about the song itself and the file creator. The specific part of the song detailing the steps of the corresponding is in the \textbf{NOTES} section. Each notes is composed of measures, where different steps in a measure are separated by the same unit of time. Each line consists of 4 integers where each digit represents a different arrow. For example the line \textbf{1001} indicates that for this step, the left and right arrow characters on a keyboard should be clicked.\\

An example of this beat information in the Stepmania file can be seen below:\\
\begin{lstlisting}
#NOTES:
     these are some sample steps:
     5:
     0.000,0.250,0.500,0.750,1.000: // 5 measures
// measure 1
2010
0000
0100
0000
, // measure 2
...
\end{lstlisting}

In the context of our project, we seek to detect beats for a given song and generate a step file from this information. As well, since Stepmania files don't have a step for each beat, we implement a heuristic function that predicts if a beat is a step based on the time of the last step and the amount of beats that has occurred since the last step.\\




\subsection{Beat Detection}
Beat detection has been investigated by other groups, usually from an audio processing perspective. Most other groups have focused on finding a consistent pattern in the frequency of beats with the intention of extracting a beats per minute value. Values associated with human hearing can be used for approximation.

Humans hearing has a temporal masking of about three miliseconds \cite{temporalmasking}. This was evaluated by playing a white noise and pausing the sound for a short amount of time to detect the longest range the pause could be before humans noticed it. This experiment found that this value varied with the pitch of the white noise being played and if the length of pause was increasing or decreasing. For the sake of this report the lowest possible value found by the report will be used and the human temporal masking threshold. This is the value at which data can be aggregated without a noticeable difference in sound. It also represents the finest granularity changes in a song can be heard. If a change in pitch or volume happens within three milliseconds of another change it will not be detected. This project aimed to keep its granularity in the range of three milliseconds to accommodate this value.

Human hearing is sensitive to a wide range of frequencies, roughly from 20Hz to 20 000Hz\cite{frequencylimits}. Music tends to only inhabit the 100Hz to 4 000Hz. This is also the range of frequencies that human hearing is most sensitive to \cite{frequencylimits}. This project put emphasis on lower frequency sounds in its frequency related features due to the fact that humans tend to hear more beats on lower frequency sounds. Within these frequency ranges humans can listen without discomfort and hear most accurately with the ability to hear a difference in tones separated by as little as 3Hz \cite{frequencylimits}. These values heavily influenced how bandwidth ranges were divided.


\section{Solution}
This system feeds a song and its metadata first into a feature extractor. This runs simple signal processing techniques to extract features about each sample in the song that more closely relate to real world values that might influence the way a human would hear it. It is then fed into a step file parser. This combines the song data is the stepmania data to label each piece of data for training. This labeled data is then used to train the neural network. The neural network determines where beats should go. This data is finally fed into a step file creator that translates the labeled data into a format that can be read by the stepmania game.
\begin{center}
	\includegraphics[scale=0.55]{data-flow.png}
\end{center}

\subsection{Feature Extraction}
\input{solution-features}

\subsection{Multilayer Perceptron}



\subsection{Outline}

The initial solution is to implement a simple MLP neural network which will be trained by raw input data. The intention is to observe the output of the trained network and the network itself and analyse the results to get a sense of direction.\\

\includegraphics[scale=0.55]{initial_approach_outline.png}

\subsubsection{Input Extraction}

The input of the system is pairs of files that compose a DDR game. One of the files is a music file and the other is a steps file. The DDR emulator will parse these 2 files and play the music file in sync with the steps file so the user may play the game.
The music file may come in many different formats such as MP3s, WAVs, AIFFs, and many more. However, all formats are just different representations of raw sound signals and can be readily converted to each other. Some file formats are compressed and will need to be uncompressed. The Wav file format is a standard uncompressed file format. This project converts all input music files to the Wav file format so that only one parser needs to be written. As well, the Wav file format is the easiest to work with because of its very transparent way of data storage.\\

The Wav file format [1] stores a frequency and bitrate followed by a series of sound signals that compose the sounds to be played. The frequency determines the time duration of each sound signal and the bitrate determines how many bits are used to represent each sound signal. The higher these values are, the finer the sound and the more it resembles analog sound. A parser is built to parse a wav file into memory as a list, where each item of the list will represent the amplitude of the sound signal at the time index * (1 / frequency). The wav file may contain more than one channels in which case the resulting list will have one dimension for each channel.
The steps file comes in only one file format (SM file[2]) for the PC emulator. The steps file represent the step data using measures. Each measure will contain data about which arrows will be displayed on the screen at each fraction of the measure (valid fractions are all over powers of 2). The length of each measure is specified in the header of the file, along with other various data that dictate timing factors. A parser is created to parse the steps file into memory as a list, where each item of the list will be a 1 or 0 to represent whether or not a step was present at the exact moment specified by the index. The amount of time each index represents is configurable and will be called with parameters to match that of the sound signal.\\

\includegraphics[scale=0.55]{signal_1.png}

Figure x. A visualisation of the sound signal array (top) and the beat signal array (bottom) of a sample input set.


\subsection{Neural Network}

To build a model for predicting beats for specific timeframes, two neural networks were used. The first neural network a custom implemented Multilayer Perceptron network. The second was a feedforward network utilized the third party Keras library wrapping Theano and Tensorflow.

\subsubsection{Custom Implemented Neural Network}
A custom implemented MLP neural network is used to train the data. The configurable parameters are the number of layers, size of each layer, and the input/output size. The neural network is also able to export its current state to file so it can be imported for further training or use.\\

As mentioned previously, the input of the neural network is the raw data of the sound signal and steps signal. A series of sound signals is expected as input and a series of steps signals is expected as output. The entire signal array for both the sound and steps are split up into chunks of 100 signals, and the input/output size of the neural network is configured to 100. For a given song, one epoch consists of running each chunk of 100 signals of the song through the neural network and performing backpropagation.
The algorithm used in the MLP neural network is identical to the backpropagation algorithm discussed in lectures.\\

 At the start of each epoch, each node in the network is initialised with a random weight for each of its parent nodes. When forward propagation of data occurs, the value of the input nodes will be initialised as a chunk of 100 signals of the song. The value of other nodes is then calculated as the summation of the product of each parent node’s value and its respective weight in the current node. The value is then put through a sigmoid activation function before being used by the child nodes. The value of the output nodes is compared to the chunk of 100 signals of the steps. Error is calculated using gradient descent and back propagated. The cumulative error is calculated as the summation of (expected output – actual output) * (1 – output) * (output) across all iterations of the epoch. The weights are updated based on errors and another epoch begins. \\\\
 
\subsection{Keras Library Network Model}

The neural network constructed with Keras was a single layer network with the following configuration:

\begin{itemize}
	\item The input vector was an array of 7 floating point numbers. Each represented a timeframe and the features extracted from it.
	\item One hidden layer with 200 nodes. This layer used a uniform initialization of weights because we had no prior knowledge as to what feature of our input vector was most important
	\item The error function used the stochastic gradient descent algorithm described in class.
	\item The learning rate was 0.01, with a decay of $1e-6$ per epoch.
	\item The output layer was a single node whose value was in $[0,1]$ because the activation function of the one hidden layer was the sigmoid function. This value represented the probability a given timeframe was a beat.
\end{itemize}
 



\subsection{Dataset Construction and Format}
\subsubsection{Dataset Format}

In order to use a multilayer perceptron network, supervised learning must be used in order to let the network learn the function you are trying to represent or approximate. For our project, the network is trained on a set of timeframes corresponding to the wavelet file for an audio track in order to detect beats. A timeframe can be defined as a portion of the audio signal for a music file. Thus, a dataset that mapped different timeframes of an audio file to a beat was required. This required researching into datasets that exposed both the metadata information for a large corpus of  audio files, and the discrete times for when beats occur.\\

With the above requirements, we could define the necessary format of our dataset. Our dataset required pairs of files for an audio track : an uncompressed wavelet file in the ‘.wav’ format and a corresponding ‘.beats’ file containing comma separated times for when beats occurred in a song.\\

\subsubsection{Music Information Retrieval Evaluation Exchange (MIREX)}

Our initial dataset was a small dataset used for a 2012 audio beat tracking competition provided by the Music Information Retrieval Evaluation Exchange (MIREX). MIREX is a contest and conference organized by the graduate school of computer science at the University of Illinois Urbana-Champaign. It provides datasets for contests attempting to identify optimal algorithms for beat detection, tempo change detection and other audio analytics. \\

The dataset used consisted of 100 songs, where each song had a file containing the times when a human audience believe beats occurred. Each audio file was in the uncompressed wavelet format, with 1 audio channel and at a standard 44.1KHz sampling rate of timeframes per second. However each song in this dataset was less than 30 seconds in length, so it did not provide enough timeframes to train any neural network model. \\

\subsubsection{Million Song Dataset}

The Million Song Dataset is a dataset containing audio analysis features for millions of songs. One of the features analyzed and available, is the discrete timing of beats. This dataset is openly available and was produced by the Echo Nost, an audio analytics company. This dataset provided all of the necessary beats information for an incredibly large corpus of songs, 280GB in size. However, unlike the Mirex dataset, it did not provide any media files that would allow us to use as training inputs to our model. Thus, we had to implement our own solution.

\subsubsection{Google Play Music}

In order to build a corpus of media files, we used the GooglePlayMusic API to to download songs. For each song we found in a subset of the MillionSongDataset, we extracted its track title, downloaded the audio file in the MPEG-2 format (mp3), and then converted it to the uncompressed wavelet format. This conversion was performed using the popular FFMPEG utility. 
\begin{center}
	\includegraphics[scale=0.55]{dataset_flow.png}
\end{center}

\section{Results}

\subsection{Method of Evaluation}

\subsection{Accuracy}

\subsection{Potential Sources of Errors}

\subsubsection{Feature Extraction}
Many of the methods for extracting features from the wave file were very naive introduced approximations and rooms for error. These could have influenced more inaccuracies than a purely signal processing approach would have. It was assumed that the introduction of a neural network would compensate for weaker feature extraction.\\

The size of the neighborhood was selected using the general value of one second. This was chosen due to its unit nature and through basic reason. It should have been a tunable parameter and configured through experimentation. The neighborhood size could have greatly effected the accuracy of the data fed into the network as it was one of the primary sources of memory within the system. It is likely that one full second was far too large as it included three hundred chunks and quite a lot can happen in one second of as song, particularly one second of a dance styled song where the beats per minute are often near 140.\\

The frequencies used to break apart the bandwidths of a chunk were taken from as paper from 1998. These values were calibrated for the author's purely signal processing approach to the problem and might not have been the best for this system's approach. Other systems that more closely resemble this one have used other frequency break points determined by musical values and tones. Following one of these methods of analyzing bandwidth values would have more closely approximated the way humans hear music.\\

The sensitivity of human hearing differs greatly at different frequencies which was not considered when evaluating the way frequency interacts with how humans hear songs. The amplitude for a given frequency was its assumed ``hearablity'' value without taking into consideration the frequency itself and how that would effect how well humans would hear it. This could have been a major source of inaccuracy in the features fed into the neural network.\\


\subsubsection{Network Values}

\section{Conclusion}

\subsection{Future Improvements}

\subsubsection{Feature Extraction}
Feature extraction was a very large portion of this project, done with perhaps the least prior knowledge. Its main goal is to most accurately represent reality and the way humans hear music and detect beats within said music. By improving on the features fed into the neural network used the accuracy of this system could greatly improve.

The features extracted for the use of this project were the most simple possible. Their accuracy could have been improved using more refined techniques. The bandwidth power variance was found by summing the energy across all values that fall in a frequency range. This is a very naive approach and the paper referenced for finding the bandwidth break values using much more advanced signal processing techniques to implement a filter bank with these values. This was pointed to as a potential source of inaccuracy, improving on these would greatly increase the relevancy of the features extracted from the input used.

Most songs have different sections where the tone is completely. This fact was not leveraged by this system at all. In a more complex system this could be leveraged this by estimating breaks between chorus, verse, and bridge to only compare values within those sections. This would allow more accurate data to be extracted by comparing local values to a much larger neighborhood.

Music is most usually described using genres. This data is downloaded with the song as part of the data pipeline. The genre a song is classified can be used to give an estimated value for its beats per minute which can be used to roughly estimate the gap between beats. This value would greatly help with the problem of many beats occurring in sequence. It would also allow the system to more accurately represent how humans hear a song by using the tags assigned to the song as reference.


\subsubsection{Recurrent Networks}

\subsubsection{Fuzzy Matching}


\newpage
\bibliographystyle{unsrt}
\bibliography{report}


\end{document}
