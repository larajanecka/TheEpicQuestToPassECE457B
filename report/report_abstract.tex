\subsection{Introduction}

Introduction
The project proposed by the following abstract falls under Cluster 2, Pattern Recognition and
Perception. The goal of Unlimited DR is to produce software that is able to generate DDR game
files using neural networks and fuzzy inferencing techniques.
DDR (Dance Dance Revolution) is a video game under the genre of rhythm games. In the game,
the player is presented with music and a sequence of arrows which they must press at an exact
time. Typically, the game is played on a foot-pad where the player presses 4 arrows with their
feet, which is akin to dancing to the music. Currently, DDR game files (SM files) are generated
manually by semi-professional or professional step makers. The step makers create the SM files
by listening to the song and creatively conjuring a set of steps to match the music; the better
the game steps matches the music, the more exciting the song is to play.


\subsection{Motivation}
The main problem with DDR as a game is the lack of SM files for a wide selection of songs. This
is due to the fact that to generate each SM file, a step maker will need to spend around 4 hours.
As the current demand for the game is not very high, the selection of songs for which step files
exists is very low (compared to number of existing songs). This is a major blocker for new
players to the game. A small study (outside the scope of this course) has been conducted to see
why people are not willing to try DDR or why people new to DDR have quit reveals that the one
driving factor to their patronage to the game is being able to play songs they love. Hence, if
software could be created to produce SM files from arbitrary songs, or arbitrary songs from a
genre set, it can easily revitalize the DDR community.

\subsection{Existing Data}
The data used for this project will be existing pairs of song to step file mappings, which were
done by the DDR community in the past. Substantial data repositories exist online and are free to
download (and play). A simple web scraping strategy can be used to obtain the data to use for the
project.

\subsection{Project Strategy}
Music files can always be decompressed to a WAV file format. A WAV file holds a sequence of
integers which each denote a single sample amplitude of the music’s waveform. The project
implementation will first attempt to generate a set of music feature candidates by applying 
various Digital Signal Processing techniques and perhaps even some randomized (but guided)
functions to arbitrary samples of the waveform. Examples of music feature candidates include
possible beats, note changes, or any seemingly relevant patterns found when processing the
amplitude samples. The purpose of the music candidate feature generation is to create a broad set
of possible music features that the step file makers used to create the step file. Fuzzy logic will
be implemented to determine whether or not a music feature candidate is indeed a relevant music
feature used to generate the steps file. A neural network will be used to facilitate the evaluation
of multiple features simultaneously, propagating the details of the feature if it is promising. The
neural network implementation is also expected to compare music feature candidates and
categorize them using fuzzy logic. The resulting neural network will then contain a rough
relationship of which music features found in a WAV file relates to steps in the steps file. It can
be then used to generate step files for a new WAV file.

\subsection{Drawbacks}

Potential drawbacks include having long algorithm running times. It is well known that some
DSP algorithms are slow (especially when operating over long audio samples like a song) and
adding feature candidate extracting will only increase run time. As well, this exact topic is not
well explored in the industry and overwhelming success has not yet been reported.